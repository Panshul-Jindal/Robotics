\documentclass[11pt]{beamer}
\usetheme{Madrid}
\usefonttheme{serif}
\usepackage[utf8]{inputenc}
\usepackage[T1]{fontenc}
\usepackage{amsmath}
\usepackage{amsfonts}
\usepackage{media9}
\usepackage{amssymb}
\usepackage{graphicx}
\usepackage{booktabs}

\setbeamertemplate{caption}[numbered]
\setbeamertemplate{navigation symbols}{}

\author[  ]{AI4000 Robotics \\ by Dr. R. Prasanth Kumar}
\title{PUMA 560 Manipulator Project}
\institute[]{Indian Institute Of Technology, Hyderabad}
\date{November 7th, 2025}

\bibliographystyle{apalike}

\begin{document}

%----------------------------------------------------------
\begin{frame}
\titlepage
\end{frame}

%----------------------------------------------------------
\begin{frame}{Outline}
\tableofcontents
\end{frame}

%----------------------------------------------------------
\section{Introduction}

\begin{frame}{Introduction}
\begin{itemize}
    \item The \textbf{PUMA 560 (Programmable Universal Machine for Assembly)} is a six-degree-of-freedom industrial manipulator.
    \item Goal: Develop a complete mathematical model including:
    \begin{itemize}
        \item Forward and inverse kinematics
        \item Dynamics
        \item Trajectory and path planning
    \end{itemize}
    \item The manipulator will execute a simple pick-and-place task in a physics-based simulator (e.g., MuJoCo).
    \item This presentation focuses on the \textbf{Forward Kinematics} derivation using the Denavit-Hartenberg (DH) convention.
\end{itemize}
\end{frame}

%----------------------------------------------------------
\section{The PUMA 560 Manipulator}

\begin{frame}{The PUMA 560 Manipulator}
\begin{itemize}
    \item A six-DOF serial manipulator with revolute joints.
    \item Composed of:
    \begin{itemize}
        \item Base, shoulder, and elbow - responsible for positioning.
        \item Wrist - controls end-effector orientation.
    \end{itemize}
\end{itemize}

\begin{figure}[htb]
    \centering
    \includegraphics[width=0.4\textwidth]{PUMA 560.png}
    \caption{PUMA 560 manipulator}
\end{figure}
\end{frame}

%----------------------------------------------------------
\section{Schematic and Coordinate Frames}

\begin{frame}{Schematic and Coordinate Frames}
Coordinate frames are assigned according to the \textbf{Denavit–Hartenberg} convention.

\begin{figure}[htb]
    \centering
    \includegraphics[width=0.46\textwidth]{sv_with_Axis.png}
    \caption{Assigned frames for the PUMA 560 arm}
\end{figure}
\end{frame}

\begin{frame}{Schematic (Wrist Section)}
\begin{figure}[htb]
    \centering
    \includegraphics[width=0.9\textwidth]{sv_with_axis_2.png}
    \caption{Assigned coordinate frames for the wrist joints}
\end{figure}
\end{frame}

\begin{frame}{Schematic (Wrist Section)}
\begin{figure}[htb]
    \centering
    \includegraphics[width=0.7\textwidth]{better_view_of_wrist.png}
    \caption{Simplified schematic of wrist joint arrangement}
\end{figure}
\end{frame}

%----------------------------------------------------------
\section{Denavit-Hartenberg Parameters}

\begin{frame}{Denavit-Hartenberg Parameters}
\begin{itemize}
    \item The DH convention defines each link using:
    \[
    (a_i, \alpha_i, d_i, \theta_i)
    \]
    \item The following table shows the DH parameters for the PUMA 560 manipulator.
\end{itemize}

\begin{table}[htb]
\centering
\caption{Denavit–Hartenberg parameters of PUMA 560}
\begin{tabular}{@{}ccccc@{}}
\toprule
Link $i$ & $\alpha_{i}$ (deg) & $a_{i}$ (m) & $d_i$ (m) & $\theta_i$ (rad) \\ \midrule
1 & $90$     & $a_1$   & $d_1$      & $\theta_1$ \\
2 & $0$   & $a_2$   & $d_2$      & $\theta_2$ \\
3 & $-90$     & $0$ & $0$    & $\theta_3$ \\
4 & $90$   & $0$ & $d_4$    & $\theta_4$ \\
5 & $-90$    & $0$   & $0$      & $\theta_5$ \\
6 & $0$   & $0$   & $d_6$      & $\theta_6$ \\ 
\bottomrule
\end{tabular}
\end{table}
\end{frame}

%----------------------------------------------------------
\section{Homogeneous Transformation Matrices}

\begin{frame}{Transformation Matrices (1-3)}

The homogeneous transformation matrices are derived from the DH parameters.

\[
A_1 =
\begin{bmatrix}
\cos\theta_1 & 0 & \sin\theta_1 & 0 \\
\sin\theta_1 & 0 & -\cos\theta_1 & 0 \\
0 & 1 & 0 & d_1 \\
0 & 0 & 0 & 1
\end{bmatrix}
\quad
A_2 =
\begin{bmatrix}
\cos\theta_2 & -\sin\theta_2 & 0 & a_2\cos\theta_2 \\
\sin\theta_2 & \cos\theta_2 & 0 & a_2\sin\theta_2 \\
0 & 0 & 1 & d_2 \\
0 & 0 & 0 & 1
\end{bmatrix}
\]

\[
A_3 =
\begin{bmatrix}
\cos\theta_3 & 0 & -\sin\theta_3 & a_3\cos\theta_3 \\
\sin\theta_3 & 0 & \cos\theta_3  & a_3\sin\theta_3 \\
0 & -1 & 0 & 0 \\
0 & 0 & 0 & 1
\end{bmatrix}
\]

\end{frame}

\begin{frame}{Transformation Matrices (4-6)}

\[
A_4 =
\begin{bmatrix}
\cos\theta_4 & 0 & \sin\theta_4 & 0 \\
\sin\theta_4 & 0 & -\cos\theta_4 & 0 \\
0 & 1 & 0 & d_4 \\
0 & 0 & 0 & 1
\end{bmatrix}
\]

\[
A_5 =
\begin{bmatrix}
\cos\theta_5 & 0 & -\sin\theta_5 & 0 \\
\sin\theta_5 & 0 & \cos\theta_5 & 0 \\
0 & -1 & 0 & 0 \\
0 & 0 & 0 & 1
\end{bmatrix}
\]

\[
A_6 =
\begin{bmatrix}
\cos\theta_6 & -\sin\theta_6 & 0 & 0 \\
\sin\theta_6 & \cos\theta_6  & 0 & 0 \\
0 & 0 & 1 & d_6 \\
0 & 0 & 0 & 1
\end{bmatrix}
\]

\end{frame}


%----------------------------------------------------------
\section{Forward Kinematics Result}

\begin{frame}{Resulting Forward Kinematics}
\[
{}^{0}\!T_6 = T_1 \, T_2 \, T_3 \, T_4 \, T_5 \, T_6
\]
\begin{block}{Final Transformation}
\[
{}^{0}\!T_6 =
\begin{bmatrix}
R_{0}^{6} & p_{0}^{6} \\
0 & 1
\end{bmatrix}
\]
where $R_{0}^{6}$ is the rotation matrix (orientation) and $p_{0}^{6}$ is the position vector.
\end{block}
\end{frame}

\begin{frame}{Workspace of PUMA 560}
\begin{figure}[htb]
    \centering
    \includegraphics[width=0.45\textwidth]{Workspace-of-the-wrist-point-of-the-PUMA-560.png}
    \caption{Workspace of the wrist point of the PUMA 560}
\end{figure}
\end{frame}


\begin{frame}{Animated Workspace of wrist point}
\begin{itemize}
    \item Use \textbf{Okular in Linux} or \textbf{Adobe Acrobat in Windows} to view
\end{itemize}
    \centering
    \includemedia[
        width=0.8\linewidth,
        height=0.6\linewidth,
        activate=pageopen,
        addresource=puma560_workspace-3d.mp4,
        flashvars={
            source=puma560_workspace-3d.mp4
            &autoPlay=true
            &loop=true
        }
        ]{}{VPlayer.swf}
\end{frame}

%----------------------------------------------------------
\section{Conclusion}

\begin{frame}{Conclusion}
\begin{itemize}
    \item Derived the homogeneous transformation matrices for all six joints.
    \item Established the overall forward kinematic model ${}^{0}\!T_6$.
    \item This model defines the end-effector’s position and orientation as functions of six joint variables.
    \item Future work:
    \begin{itemize}
        \item Inverse kinematics
        \item Dynamics modeling
        \item Trajectory and path planning in simulation
    \end{itemize}
\end{itemize}
\end{frame}

%----------------------------------------------------------
\begin{frame}
\centering
\Huge Thank You!\\[1em]
\end{frame}

\end{document}
