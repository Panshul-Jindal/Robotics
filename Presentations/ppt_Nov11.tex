\documentclass[11pt]{beamer}
\usetheme{Madrid}
\usefonttheme{serif}
\usepackage[utf8]{inputenc}
\usepackage[T1]{fontenc}
\usepackage{amsmath}
\usepackage{amsfonts}
\usepackage{media9}
\usepackage{amssymb}
\usepackage{graphicx}
\usepackage{booktabs}

\setbeamertemplate{caption}[numbered]
\setbeamertemplate{navigation symbols}{}

\title{PUMA 560 Manipulator Project}
\subtitle{Velocity Kinematics}

\author[AI4000 Robotics by Dr. R. Prasanth Kumar]{Saran Konala \quad Aryan Mehta \\  Pakki Sai Kaushik \quad Panshul Jindal}
\institute[]{Indian Institute of Technology, Hyderabad}
\date{November 11th, 2025}

\bibliographystyle{apalike}

\begin{document}

%----------------------------------------------------------
\begin{frame}
\titlepage
\end{frame}

%----------------------------------------------------------
\begin{frame}{Outline}
\tableofcontents
\end{frame}

%----------------------------------------------------------
\section{Schematic and Coordinate Frames}

\begin{frame}{Schematic and Coordinate Frames}
Coordinate frames are assigned according to the \textbf{Denavit–Hartenberg} convention.

\begin{figure}[htb]
    \centering
    \includegraphics[width=0.46\textwidth]{sv_with_Axis.png}
    \caption{Assigned frames for the PUMA 560 arm}
\end{figure}
\end{frame}


%----------------------------------------------------------
\section{Denavit-Hartenberg Parameters}

\begin{frame}{Denavit-Hartenberg Parameters}
\begin{itemize}
    \item The DH convention defines each link using:
    \[
    (a_i, \alpha_i, d_i, \theta_i)
    \]
    \item The following table shows the DH parameters for the PUMA 560 manipulator.
\end{itemize}

\begin{table}[htb]
\centering
\caption{Denavit–Hartenberg parameters of PUMA 560}
\begin{tabular}{@{}ccccc@{}}
\toprule
Link $i$ & $\alpha_{i}$ (deg) & $a_{i}$ (m) & $d_i$ (m) & $\theta_i$ (rad) \\ \midrule
1 & $90$     & $a_1$   & $d_1$      & $\theta_1$ \\
2 & $0$   & $a_2$   & $d_2$      & $\theta_2$ \\
3 & $-90$     & $0$ & $0$    & $\theta_3$ \\
4 & $90$   & $0$ & $d_4$    & $\theta_4$ \\
5 & $-90$    & $0$   & $0$      & $\theta_5$ \\
6 & $0$   & $0$   & $d_6$      & $\theta_6$ \\ 
\bottomrule
\end{tabular}
\end{table}
\end{frame}

%----------------------------------------------------------
\section{Homogeneous Transformation Matrices}

\begin{frame}{Transformation Matrices (1-3)}

The homogeneous transformation matrices are derived from the DH parameters.

\[
A_1 =
\begin{bmatrix}
\cos\theta_1 & 0 & \sin\theta_1 & 0 \\
\sin\theta_1 & 0 & -\cos\theta_1 & 0 \\
0 & 1 & 0 & d_1 \\
0 & 0 & 0 & 1
\end{bmatrix}
\quad
A_2 =
\begin{bmatrix}
\cos\theta_2 & -\sin\theta_2 & 0 & a_2\cos\theta_2 \\
\sin\theta_2 & \cos\theta_2 & 0 & a_2\sin\theta_2 \\
0 & 0 & 1 & d_2 \\
0 & 0 & 0 & 1
\end{bmatrix}
\]

\[
A_3 =
\begin{bmatrix}
\cos\theta_3 & 0 & -\sin\theta_3 & a_3\cos\theta_3 \\
\sin\theta_3 & 0 & \cos\theta_3  & a_3\sin\theta_3 \\
0 & -1 & 0 & 0 \\
0 & 0 & 0 & 1
\end{bmatrix}
\]

\end{frame}

\begin{frame}{Transformation Matrices (4-6)}

\[
A_4 =
\begin{bmatrix}
\cos\theta_4 & 0 & \sin\theta_4 & 0 \\
\sin\theta_4 & 0 & -\cos\theta_4 & 0 \\
0 & 1 & 0 & d_4 \\
0 & 0 & 0 & 1
\end{bmatrix}
\]

\[
A_5 =
\begin{bmatrix}
\cos\theta_5 & 0 & -\sin\theta_5 & 0 \\
\sin\theta_5 & 0 & \cos\theta_5 & 0 \\
0 & -1 & 0 & 0 \\
0 & 0 & 0 & 1
\end{bmatrix}
\]

\[
A_6 =
\begin{bmatrix}
\cos\theta_6 & -\sin\theta_6 & 0 & 0 \\
\sin\theta_6 & \cos\theta_6  & 0 & 0 \\
0 & 0 & 1 & d_6 \\
0 & 0 & 0 & 1
\end{bmatrix}
\]

\end{frame}


%----------------------------------------------------------
\section{Forward Kinematics Result}

\begin{frame}{Resulting Forward Kinematics}
\[
{}^{0}\!T_6 = T_1 \, T_2 \, T_3 \, T_4 \, T_5 \, T_6
\]
\begin{block}{Final Transformation}
\[
{}^{0}\!T_6 =
\begin{bmatrix}
R_{0}^{6} & p_{0}^{6} \\
0 & 1
\end{bmatrix}
\]
where $R_{0}^{6}$ is the rotation matrix (orientation) and $p_{0}^{6}$ is the position vector.
\end{block}
\end{frame}

\begin{frame}{Workspace of PUMA 560}
\begin{figure}[htb]
    \centering
    \includegraphics[width=0.45\textwidth]{Workspace-of-the-wrist-point-of-the-PUMA-560.png}
    \caption{Workspace of the wrist point of the PUMA 560}
\end{figure}
\end{frame}


\begin{frame}{Animated Workspace of wrist point}
\begin{itemize}
    \item Use \textbf{Okular in Linux} or \textbf{Adobe Acrobat in Windows} to view
\end{itemize}
    \centering
    \includemedia[
        width=0.8\linewidth,
        height=0.6\linewidth,
        activate=pageopen,
        addresource=puma560_workspace-3d.mp4,
        flashvars={
            source=puma560_workspace-3d.mp4
            &autoPlay=true
            &loop=true
        }
        ]{}{VPlayer.swf}
\end{frame}
%----------------------------------------------------------
\section{Velocity Kinematics}
\begin{frame}{Kinematics: Joint Axes (${}^{0}\mathbf{z}_i$)}
\begin{block}{Axes in Base Frame}
\tiny
\[
\begin{aligned}
{}^{0}\mathbf{z}_0 &= 
\begin{bmatrix}0\\0\\1\end{bmatrix},
\qquad
{}^{0}\mathbf{z}_1 = 
\begin{bmatrix}s_1\\-c_1\\0\end{bmatrix},
\qquad
{}^{0}\mathbf{z}_2 = 
\begin{bmatrix}s_1\\-c_1\\0\end{bmatrix},
\\[1.0em]
{}^{0}\mathbf{z}_3 &= 
\begin{bmatrix}
-\,c_1\,s_{23} \\[3pt]
-\,s_1\,s_{23} \\[3pt]
\phantom{-}\,c_{23}
\end{bmatrix},
\\[1.0em]
{}^{0}\mathbf{z}_4 &= 
\begin{bmatrix}
-\,s_1 c_4 + c_1 c_{23} s_4 \\[4pt]
\phantom{-}\,s_1 c_{23} s_4 + c_1 c_4 \\[4pt]
-\,s_{23} s_4
\end{bmatrix},
\\[1.0em]
{}^{0}\mathbf{z}_5 &= 
\begin{bmatrix}
- (s_1 s_4 + c_1 c_4 c_{23}) s_5 - c_1 s_{23} c_5 \\[4pt]
- (c_1 s_4 - s_1 c_4 c_{23}) s_5 - s_1 s_{23} c_5 \\[4pt]
\phantom{-}\,s_{23} s_5 c_4 - c_{23} c_5
\end{bmatrix},
\\[1.0em]
{}^{0}\mathbf{z}_6 &= {}^{0}\mathbf{z}_5.
\end{aligned}
\]
\end{block}
\end{frame}


\begin{frame}{Kinematics: Frame Origins (${}^{0}\mathbf{o}_i$)}
\begin{block}{Origins in Base Frame}
\tiny
\[
\begin{aligned}
{}^{0}\mathbf{o}_0 &= 
\begin{bmatrix}0\\0\\0\end{bmatrix},
\qquad
{}^{0}\mathbf{o}_1 = 
\begin{bmatrix}0\\0\\d_1\end{bmatrix},
\\[1.0em]
{}^{0}\mathbf{o}_2 &= 
\begin{bmatrix}
a_2 c_1 c_2 + d_2 s_1 \\[3pt]
a_2 s_1 c_2 - d_2 c_1 \\[3pt]
a_2 s_2 + d_1
\end{bmatrix},
\\[1.0em]
{}^{0}\mathbf{o}_3 &= 
\begin{bmatrix}
c_1(a_2 c_2 + a_3 c_{23}) + d_2 s_1 \\[4pt]
s_1(a_2 c_2 + a_3 c_{23}) - d_2 c_1 \\[4pt]
d_1 + a_2 s_2 + a_3 s_{23}
\end{bmatrix},
\\[1.0em]
{}^{0}\mathbf{o}_4 &= 
\begin{bmatrix}
c_1(a_2 c_2 + a_3 c_{23}) + d_2 s_1 - d_4 c_1 s_{23} \\[4pt]
s_1(a_2 c_2 + a_3 c_{23}) - d_2 c_1 - d_4 s_1 s_{23} \\[4pt]
d_1 + a_2 s_2 + a_3 s_{23} + d_4 c_{23}
\end{bmatrix},
\\[1.0em]
{}^{0}\mathbf{o}_5 &= {}^{0}\mathbf{o}_4,
\\[1.0em]
{}^{0}\mathbf{o}_6 &= 
{}^{0}\mathbf{o}_5 + d_6\,{}^{0}\mathbf{z}_5.
\end{aligned}
\]
\end{block}
\end{frame}


\begin{frame}{Spatial Manipulator Jacobian}
\vspace{-0.2cm}
\begin{block}{Jacobian Structure}
The spatial velocity Jacobian $\mathbf{J}$ is a $6 \times 6$ matrix composed of:
\begin{itemize}
    \item \textbf{Linear Velocity $\mathbf{J}_v$}: Contribution of joint $i$ to end-effector's linear velocity
    \item \textbf{Angular Velocity $\mathbf{J}_\omega$}: Contribution of joint $i$ to end-effector's angular velocity
\end{itemize}
\end{block}

\begin{block}{Column Formulas for Revolute Joint $i$}
\begin{align*}
\mathbf{J}_{v_i} &= {}^{0}\mathbf{z}_{i-1}\times\!\left({}^{0}\mathbf{o}_6-{}^{0}\mathbf{o}_{i-1}\right) \\
\mathbf{J}_{\omega_i} &= {}^{0}\mathbf{z}_{i-1}
\end{align*}
\end{block}
\end{frame}

\begin{frame}
\frametitle{Full $6 \times 6$ Spatial Jacobian $\mathbf{J}$}

% Tighter spacing
\setlength{\arraycolsep}{3pt}
\renewcommand{\arraystretch}{1.1}
\footnotesize

\begin{block}{Structure}
\[
\mathbf{J} = 
\begin{bmatrix}
\mathbf{J}_v \\[4pt]
\mathbf{J}_\omega
\end{bmatrix}
\]
\end{block}

\begin{block}{Linear Velocity Component $\mathbf{J}_v$}
\centering
\resizebox{\linewidth}{!}{$
\mathbf{J}_v =
\begin{bmatrix}
\;d_2 c_1 + d_4 s_1 s_{23}
 - (a_2 c_2 + a_3 c_{23}) s_1
 + d_6\!\left[(s_1 c_4 c_{23} - s_4 c_1)s_5 + s_1 s_{23} c_5\right]
&
c_1\!\left[-a_2 s_2 - a_3 s_{23} - d_4 c_{23}
            + d_6(s_5 s_{23} c_4 - c_5 c_{23})\right]
&
c_1\!\left[-a_3 s_{23} - d_4 c_{23}
            + d_6(s_5 s_{23} c_4 - c_5 c_{23})\right]
&
d_6(-s_1 c_4 + s_4 c_1 c_{23}) s_5
&
- d_6\!\left[(s_1 s_4 + c_1 c_4 c_{23}) c_5 - s_5 s_{23} c_1\right]
&
0
\\[10pt]
%
d_2 s_1 - d_4 s_{23} c_1
 + (a_2 c_2 + a_3 c_{23}) c_1
 + d_6\!\left[(s_1 s_4 + c_1 c_4 c_{23}) s_5 - s_{23} c_1 c_5\right]
&
- s_1\!\left[a_2 s_2 + a_3 s_{23} + d_4 c_{23}
           + d_6(s_5 s_{23} c_4 + c_5 c_{23})\right]
&
- s_1\!\left[a_3 s_{23} + d_4 c_{23}
           + d_6(s_5 s_{23} c_4 + c_5 c_{23})\right]
&
- d_6(s_1 s_4 c_{23} + c_1 c_4) s_5
&
d_6\!\left[(s_1 c_4 c_{23} - s_4 c_1) c_5 + s_1 s_5 s_{23}\right]
&
0
\\[10pt]
%
0
&
a_2 c_2 + a_3 c_{23} - d_4 s_{23}
 + d_6(s_5 c_4 c_{23} + s_{23} c_5)
&
a_3 c_{23} - d_4 s_{23}
 + d_6(s_5 c_4 c_{23} + s_{23} c_5)
&
- d_6 s_4 s_5 s_{23}
&
d_6(s_5 c_{23} + s_{23} c_4 c_5)
&
0
\end{bmatrix}
$}
\end{block}


\begin{block}{Angular Velocity Component $\mathbf{J}_\omega$}
\centering
\resizebox{\linewidth}{!}{$
\mathbf{J}_\omega =
\begin{bmatrix}
0 & s_1 & s_1 & - c_1 s_{23} & - s_1 c_4 + s_4 c_1 c_{23} & - (s_1 s_4 + c_1 c_4 c_{23}) s_5 - c_1 s_{23} c_5
\\[6pt]
0 & -c_1 & -c_1 & - s_1 s_{23} & s_1 s_4 c_{23} + c_1 c_4 & (s_1 c_4 c_{23} - s_4 c_1) s_5 - s_1 s_{23} c_5
\\[6pt]
1 & 0 & 0 & c_{23} & - s_4 s_{23} & s_5 s_{23} c_4 - c_5 c_{23}
\end{bmatrix}
$}
\end{block}

% \end{frame}


\medskip
\scriptsize
\textbf{Notation: } $s_i=\sin\theta_i,\; c_i=\cos\theta_i,\; s_{23}=\sin(\theta_2+\theta_3),\; c_{23}=\cos(\theta_2+\theta_3)$
\end{frame}


%----------------------------------------------------------
\section{Conclusion}

\begin{frame}{Conclusion}
Future work:
    \begin{itemize}
        \item Inverse kinematics
        \item Dynamics modeling
        \item Trajectory and path planning in simulation
    \end{itemize}
\end{frame}

%----------------------------------------------------------
\begin{frame}
\centering
\Huge Thank You!\\[1em]
\end{frame}

\end{document}
